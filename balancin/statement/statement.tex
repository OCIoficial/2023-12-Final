\documentclass{oci}
\usepackage[utf8]{inputenc}
\usepackage{lipsum}

\title{Balancín}

\begin{document}
\begin{problemDescription}
    Camila y Gabriel fueron al parque a jugar a su juego favorito: el balancín.
    En este emocionante juego de coordinación y equilibrio, dos personas se
    sientan en mitades opuestas de una tabla con un eje de rotación central.
    %
    Ambos participantes ocupan su peso para ejercer fuerza en su lado de la tabla,
    de forma que mientras uno bajo el otro sube, ocasionando así un movimiento
    continuo hacia arriba y hacia abajo.
    %
    !`Es divertidísimo!

    Para que el balancín se mueva adecuadamente, ambas personas deben ejercer
    la misma fuerza en cada lado.
    %
    La fuerza que ejerce una persona depende de su peso y de su distancia
    al centro del balancín.
    %
    Específicamente, la fuerza se calcula como $peso \times distancia$.

    Para poder jugar, Camila y Gabriel deben buscar una forma de ejercer
    la misma fuerza.
    %
    Ellos no pueden cambiar su peso, pero sí pueden elegir en qué parte
    del balancín se sientan.
    %
    El balancín del parque tiene $n$ asientos en cada lado,
    estos son numerados del $1$ al $n$,
    y el $i$-ésimo asiento está a una distancia $i$ del centro.

    Por ejemplo, si los pesos de Camila y Gabriel son $50$ y $30$
    respectivamente, y el balancín tiene 10 asientos por lado,
    Camila puede sentarse en la posición $6$ y Gabriel en la $10$.
    %
    De esta forma, la fuerza que ejercerá cada uno será $50*6=300=30*10$,
    y como son iguales, podrán jugar.

    Ambos están muy emocionados por comenzar, pero no están seguros de si
    existe una forma de sentarse que les permita jugar, por lo que te pidieron
    que lo averigües por ellos. ¿Podrás ayudarlos?

\end{problemDescription}

\begin{inputDescription}
    La entrada consiste en una línea con tres enteros $n, c, g$
    $(1 \leq n, c, g \leq 10^9)$ correspondientes respectivamente
    a la cantidad de asientos del balancín, el peso de Camila
    y el peso de Gabriel.
\end{inputDescription}

\begin{outputDescription}
	Debes imprimir \texttt{SI} en caso de existir una forma de
    que Camila y Gabriel se sienten en el balancín tal que puedan jugar,
    y \texttt{NO} en caso contrario.
\end{outputDescription}

\begin{scoreDescription}
	\subtask{15} Se probarán varios casos de prueba donde $1 \leq n, c, g \leq 10^3$.
	\subtask{25} Se probarán varios casos de prueba donde $1 \leq n, c, g \leq 10^6$.
	\subtask{60} Se probarán varios casos de prueba sin restricciones adicionales.
\end{scoreDescription}

\begin{sampleDescription}
\sampleIO{sample-1}
\sampleIO{sample-2}
\end{sampleDescription}

\end{document}
